\chapter{总结与展望}
本文围混合集成自脉冲DFB激光器的设计与制作展开,制作了一个基于硅基III-V混合集成的自脉冲DFB激光并研究了其在三个领域的应用。第一个是微波光子学领域,本文研究了利用自脉冲激光器产生光学微波信号及其频率锁定现象。第二个是光互联领域,自脉冲DFB激光器可以利用光子共振现象提升调制带宽。第三个是气体浓度检测领域,利用自脉冲DFB激光器的双波长特性,一个波长位于气体的吸收峰上,另一个波长位于吸收峰附近但不被吸收作为参考,经过对比可以得到气体的浓度信息。本论文的主要工作具体如下三个部分:

\begin{enumerate}[(1)]
	\item 
	研究了SOI芯片在刻蚀过程中刻蚀深度与颜色的对应关系,通过仿真计算得到了SOI芯片硅层在不同厚度情况下的反射谱,并利用光谱与颜色的对应关系,得到了SOI芯片不同硅层厚度的色谱图。研究发现,在硅层厚度为120~$nm$以下时,SOI芯片的颜色随厚度变化非常明显。本文在实验上进行了验证,该方法可用于快速确定SOI芯片上硅层的厚度,为超净室工艺制作提供了便利。
	\item 
	提出了一种混合集成自脉冲激光器的方案,利用两段式DFB结构,实现了脉冲频率连续可调的激光器。该激光器可用于全光网络中的时钟信号,也可以用来产生光学微波信号。本文还研究了自脉冲信号与电流和温度之间的关系,还发现了其微波信号频率锁定的现象,得到了目前报道最小的锁定功率-17 dBm,可以实现带宽小于10~Hz的光学微波信号。该种设计还可以利用光子共振现象来提升激光器的调制带宽,我们通过测试该激光器的S\SB{21}参数进行了研究,测到该激光器的调制带宽可以达到23 GHz左右,并做了数据传输实验,实现了45 Gbps的数据传输速率。
	\item 
	设计了一套利用自脉冲激光器与光谱仪结合的气体浓度检测系统。光谱仪采用基于EDG的设计,研究了密集阵列波导阵列在EDG输出波导中的应用,利用其低串扰的特性,提高了EDG光谱仪的分辨率。使用该设计,可以在3~$mm$~$\times$~3~$mm$的器件尺寸实现121通道,通道间隔为0.5 $nm$,测量带宽为60 $nm$的光谱仪。本文在EDG反射光栅位置的设计上使用了具有两个完美成像点的设计方法,并通过扫描绝对光程,可以使得边缘通道的性能得到保障。实验上我们采用了描边法部分克服了EBL 150\SP{TWO}系统拼接模式与FBMS模式之间的错位问题,实现了20通道的实验结果,通道间隔为0.5 $nm$,通道的3 dB带宽为0.28 $nm$,光谱分辨率($\lambda/\Delta\lambda$)为5571,插入损耗约为6.9~dB,最大串扰为-4.3~dB,通道不均匀性为1.7~dB。
	
\end{enumerate}

限于作者的精力与实验条件,以上工作还可以进一步完善和研究:

\begin{enumerate}[(1)]
	\item 
	现阶段颜色的比对只是通过肉眼来确定的,往往不够准确且受个人的影响比较大,接下去可以利用计算机通过相关算法进行颜色的匹配以增加可靠性。而且现在该方法只能用在大面积的硅层情况下,之后可以结合显微镜,将已开发的软件集成到显微镜软件中,帮助快速决定局部硅层的厚度。而且,该方法也可以推广到其他材料中去。
	\item 
	之前由于设备的原因,该激光器产生的微波信号只测到40 GHz。可以利用太赫兹波检测设备,来研究该激光器在太赫兹波产生方面的应用。鉴于目前5G研究领域非常火热,该自脉冲激光器可以产生相应的微波信号,可以将该激光器的应用领域扩展到5G当中去。
	\item 自脉冲激光器的结构参数还可以进一步优化来提升调制带宽。本文中两段DFB激光器的长度相等,可以进一步研究两段长度不相等时激光器的调制带宽。
	\item 
	对于EBL加工的通道限制问题,可以使用流片工艺进行解决,得到设计的121通道的EDG光谱仪。而且利用了流片工艺之后,通道的均匀性可以进一步提升。现阶段制作的光谱仪没有集成探测器,还需要外接探测器进行测量,之后还可以在每个通道输出口集成片上探测器,以实现真正的片上光谱仪。针对气体浓度检测系统,可以将气室利用谐振腔增加光程集成到片上,实现片上的气体浓度检测系统。
\end{enumerate}