\begin{thanks}

时光飞逝,转眼已经是我在求是园的的第十个学年了。我从当初懵懂的少年到现在即将博士毕业,收获的不仅仅是知识与技能,还有宝贵的师恩与友情。在此,我想向他们致以最诚挚的谢意。

首先,我由衷地感谢我博士期间的导师何赛灵教授。感谢何老师在科研上对我的细心指导与支持,他在科研方向上总能够做到高屋建瓴,为我指明研究的方向,每一次的交流,都让我受益匪浅。除此之外,何老师还给予了我完善的实验条件,让我的课题能够顺利进行。平日里,何老师孜孜不倦的工作态度为我树立了良好的榜样,是我一生工作学习的榜样。在生活上,何老师也对我关心有加,尤其在我延期期间,依然给我提供基本的生活保障。

感谢同课题组的戴道锌教授,戴老师具有扎实的理论功底和勤奋执着的科研态度,每一次交流总能帮我解决困惑。感谢同课题组的时尧成教授,时老师具有丰富的微纳加工工艺经验,每当我在实验上遇到困难的时候,他总能够帮我找到解决的途径。

感谢超净室实验员胡鑫松师傅在实验上对我的帮助,胡师傅对超净室的设备非常熟悉,每当设备遇到故障他总能够在第一时间帮助我们将设备修好,为我们的实验保驾护航。感谢超净室另一位实验员陈辉,他风趣幽默的性格为我在超净室的枯燥时光增添了许多乐趣,兢兢业业的工作态度为我的实验提供了不少便捷。

感谢COER团队每一位帮助过我的同学,特别需要感谢PLC组已经毕业的黄强盛博士,感谢他带我进入了科研的大门,并在理论与实验技能方面的指导。感谢已经毕业的刘清坤博士、陈鹏鑫博士、徐培鹏博士、付鑫博士、王健博士、孙耀然博士、金里博士、于龙海博士、陈思涛博士、张森林博士、张宇光博士、林宏泽博士、董泳江博士、张磊、王晓坤、彭伟、毛毛、郑佳久、刘红燕、马可,感谢和我一起奋斗的吴昊、韩守保、王楠、刘鹏浩、张健豪、李莹、黄圣哲、姜玮、迟克群、杨雄,感谢PLC组的师弟师妹们:张明、王世鹏、尹延龙、李江、郭庭彪、陈敬业、李晨蕾、谭莹、刘卫喜、宋立甲、李晨曦、贾婧、严家林、单海峰、喻佳、张龙、刘大建、刘二虎、江小辉、潘炳呈、刘超越、高严、叶超超、董鹏辉、刘杨、丁明飞、项宇銮等。谢谢你们陪我度过了充实的博士求学时光。

感谢母校浙江大学给我提供的学习机会,给我的学生生涯留下了宝贵的精神财富,唯有今后在工作岗位上用所学知识奉献回报社会,才能不愧作为一名浙大人。

感谢根特大学的老师和同学,当我在比利时交流期间,对我科研,实验和生活上无私的帮助和指导。感谢你们让我在博士期间体验了异国风味,给我带来生活和科研上的乐趣,并且给予我崭新的科研视角。他们是:Geert Morthier, Mahmoud Shahin, Amin Abbasi, Michael Vanslembrouck, Steven Verstuyft, Liesbet Van Landschoot, 王哲超、厉彦璐、叶楠、谢卫强、朱云鹏、贾小宁、王瑞军、胡琛、张菁、赵浩澜、李昂、邢宇飞、刘旭老师、梁宇鑫、陈冠宇、朱静浩、刘静、张贺阳等。

我还要感谢我的父母、二舅以及在天国的外婆对我的养育之恩,谢谢您们对我的支持、理解和默默的付出,使我能够健康的成长成才,您们是我坚强的后盾。

最后我要衷心感谢我的女朋友卢梦娇对我科研上的鼓励与支持,伴我度过最艰难的一段求学时光,这让我对我们的未来充满期待,一起去追寻诗与远方。


\begin{flushright}
	马珂奇
	
	2019年4月于启真湖畔
\end{flushright}
\end{thanks}
