\begin{abstract}

随着人们对通信带宽的需求越来越大,光通信技术作为唯一可以提供解决方案的技术获得了飞速的发展。大数据时代的到来,光互联技术也开始从提供长距离的数据传输服务迈向机柜与机柜之间,甚至到芯片内部。此时传统的分立光器件已经不能满足实际的需求,集成光学应运而生。硅基光电子集成技术,由于具有与传统半导体产业中的互补金属氧化物半导体(Complementary Metal Oxide Semiconductor, CMOS)工艺相兼容的巨大成本优势,成为最具前景的技术。但是由于硅材料的间接带隙特性,其发光效率特别低,硅基III-V混合集成技术作为解决硅基芯片上的光源问题成为了一个热门的研究课题。由于已经投入了海量的资金,人们不仅仅希望只将集成光电子技术的应用领域局限在光互联领域,而是能够在其他更多领域发挥集成光电子技术的潜在应用。本文主要针对硅基III-V混合集成DFB激光器进行了设计与制作,并研究了其在微波光子学、光互联、气体浓度检测领域的应用。针对气体浓度检测领域,本文还设计制作了一种基于EDG的片上高分辨率光谱仪,结合自脉冲激光器设计了一套可以实现气体浓度检测的系统。

首先,本文介绍了平面光波导的基本理论和本文需要用到的数值计算方法。之后,本文介绍了DFB激光器的基本原理,并通过计算速率方程得出了其直调带宽与弛豫振荡频率之间的关系。此外还介绍了超净室中制备光电子器件常用的工艺流程,包括芯片清洗与匀胶、电子束曝光、干法与湿法刻蚀。

然后,本文根据集成光电子器件制备过程中常需要对SOI芯片进行减薄处理的情况,通过对不同硅层厚度的SOI芯片进行数值仿真,首次得出了不同硅层厚度对应的颜色谱。通过辅助软件,该颜色谱能够在超净室制备集成器件过程中帮助快速确定硅层厚度。通过实验验证,当硅层厚度小于120~$nm$时,可以达到较高的测量精度。

接着,本文提出了一种硅基混合集成自脉冲DFB激光器的方案,利用两段式DFB结构,通过拍频实现了脉冲频率连续可调的激光器。通过调节泵浦电流,微波信号的频率可以实现从5~\~{}40~GHz连续可调。本文还分析了泵浦电流与温度对微波信号频率的影响。该自脉冲激光器可用于全光网络中的时钟信号,也可以用来产生光学微波信号。本文还发现,当泵浦电流信号中包含微波信号时,激光器输出的光学微波信号会出现频率锁定的现象,从而可以实现线宽小于10~Hz的光学微波信号,本文实现了微波功率目前报道最低的-17~dBm的频率锁定。该激光器还可以利用光子共振现象来提升激光器的调制带宽到23~GHz左右,最终可以实现45~Gbps的数据传输速率。

最后,本文提出了一套利用自脉冲激光器的双波长特性与光谱仪结合的气体浓度检测系统,光谱仪采用基于EDG的方案。本文设计并优化了121通道的EDG光谱仪,分别从波导阵列和反射光栅两个方面进行了优化。波导阵列方面本文采用了模式复用中经常用到的密集阵列波导,第一次将其应用到EDG中,将本来SOI平台上间距2.5 $\mu m$、5 $\mu m$的波导间距缩小到1 $\mu m$,大大增加了EDG的光谱分辨率。在反射光栅方面,本文采用DBR反射镜,对其位置用两个完美成像点的方法进行了优化,并通过扫描绝对光程,提升了边缘通道的性能。之后本文根据设计进行了该光谱仪的工艺制作,受限于本实验室仪器的加工能力,只制作了20通道的EDG光谱仪。在实验过程中,本文采用了一种称为EBL描边法的加工工艺。最后测得EDG光谱仪的通道间隔为0.5~$nm$,通道的3 dB带宽为0.28$ nm$,光谱分辨率($\lambda/\Delta\lambda$)为5571,插入损耗约为6.9~dB,最大串扰为-4.3~dB,通道不均匀性为1.7~dB。


\keywords{集成光学~~硅基光电子集成技术~~微波光子学~~混合集成~~DFB激光器~~红外吸收光谱学~~气体浓度检测~~光谱仪~~刻蚀衍射光栅}
\end{abstract}
