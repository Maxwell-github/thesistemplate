\begin{englishabstract}
With the increasing demand for communication bandwidth, optical communication technology has achieved rapid development as the only technology that can provide solutions. With the advent of the era of big data, optical interconnect technology has also begun to provide data transmission services between cabinets and cabinets, and even inside chips. At this time, the traditional discrete optical devices can not meet the actual needs, and integrated photonics emerged. Silicon-based optoelectronic integration technology has become the most promising technology due to its enormous cost advantages in compatibility with Complementary Metal Oxide Semiconductor (CMOS) processes in the traditional semiconductor industry. However, due to the indirect band gap characteristics of silicon materials, the luminous efficiency is particularly low. The silicon-based III-V heterogeneously integration technology has become a hot research topic as a light source on silicon-based chips. Due to the huge amount of money invested in this field, people do not want to limit the application field of integrated optoelectronic technology to the field of optical interconnect, but also can play the potential application of integrated optoelectronic technology in many other fields. This thesis mainly focuses on designing and fabricating silicon-based III-V heterogeneously integrated DFB laser, and studies their applications in microwave photonics, optical interconnects and gas concentration detection. For the field of gas concentration detection, a high-resolution on-chip spectrometer based on EDG is designed. Using the designed spectrometer with a self-pulsating DFB laser, we design a system for gas concentration detection.

First of all, we introduce the basic theory of planar optical waveguides and the numerical calculation methods needed in this thesis. Then the basic principle of DFB laser diodes is introduced. In addition, the common process flow for fabricating optoelectronic devices in the clean room is introduced, including wafer cleaning and spinning coating of photoresist, electron beam lithography(EBL), dry and wet etching.

Secondly, according to the situation that the SOI wafer is often thinned in the fabricating process of optoelectronic devices, the color spectrum corresponding to the different thickness of silicon layer is obtained by numerical simulation. Through the auxiliary software, this color spectrum can help to quickly determine the thickness of the silicon layer during the fabrication of the integrated devices in the clean room. It is verified by experiments that when the thickness of the silicon layer is less than 120~$nm$, high measurement accuracy can be achieved.

Thirdly, a heterogeneously integrated self-pulsating DFB laser is proposed. The two-section DFB structure is used to realize a laser with continuously tunable pulse frequency. By adjusting the bias currents, the frequency of the microwave signal can be continuously tuned from 5 to 40~GHz. The effect of bias current and temperature on the frequency of microwave signal is also analyzed. The self-pulsating laser can be used for clock signals in all-optical networks and can also be used to generate optical microwave signals. We also find that when the input current contains microwave signal, the optical microwave signal output by the laser will be locked, thus realizing the optical microwave signal with linewidth less than 10~Hz. We achieve the frequency locking with the microwave power of -17 dBm, which is the lowest to our knowledge. The laser can also increase the modulation bandwidth of the laser using the photon-photon resonance phenomenon. The bandwidth of the S\SB{21} parameter can be increased to 23~GHz, and the data transmission rate of 45~Gbps can be realized.

Finally, a gas concentration detection system based on self-pulsating laser and spectrometer is proposed. The spectrometer adopts EDG-based scheme. A 121-channel EDG spectrometer is designed and optimized from two aspects: waveguide array and reflection grating. The densely packed waveguide array often used in mode multiplexing are used in EDG for the first time, which reduces a normal pitch of 2.5 $\mu m$, 5 $\mu m$ on the SOI platform to 1 $\mu m$. This greatly increases the spectral resolution of the EDG spectrometer. In terms of reflection gratings, this thesis uses DBR whose positions are optimized using two stigmatic imaging points method. The performance of the outmost channels are optimized by scanning the absolute optical path. After that, the spectrometer is fabricated according to the design. Due to the limitation of our laboratory equipment, only 20 channels of the EDG spectrometer are fabricated. In the fabrication process, a method called trace-retrace technique is adopted. The channel spacing of the fabricated EDG spectrometer is 0.5~$nm$, the 3 dB bandwidth of each channel is 0.28 $nm$, the spectral resolution ($\lambda/\Delta\lambda$) is 5571, and the insertion loss is about 6.9~dB. The maximum crosstalk is -4.3~dB and the channel non-uniformity is about 1.7~dB.

\englishkeywords{integrated photonics, silicon-based optoelectronic integration technology, microwave photonics, heterogeneously integration, DFB laser, infrared absorption spectroscopy, gas concentration detection, spectrometer, echelle grating}

\end{englishabstract}
